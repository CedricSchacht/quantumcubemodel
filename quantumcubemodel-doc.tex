\documentclass{article}
\usepackage[a4paper,margin=1in]{geometry}
\usepackage{hyperref}
\usepackage{quantumcubemodel}
\usepackage[backend=biber, style=authortitle]{biblatex}
\addbibresource{qcm.bib}

\title{Quantum Cube Model LaTeX Package (v0.1.0)}
\author{Cedric Schacht \\ \texttt{qcm@sch8.eu}}
\date{\today}

\begin{document}
\maketitle

\section*{Introduction}

The \textbf{Quantum Cube Model (QCM)} package provides commands that make it easy to create diagrams representing the quantum cube model for up to 3 qubits.
The package simplifies drawing complex quantum state diagrams inspired by Prof.~B.~Just's framework.

\section*{Requirements}

This package requires the following LaTeX packages:

\begin{verbatim}
\RequirePackage{braket}
\RequirePackage{xcolor}
\RequirePackage{tikz}
\usetikzlibrary{3d, calc, arrows.meta}
\end{verbatim}

\section*{Provided Commands}

\begin{description}
    \item Draws a diagram for a \textbf{single qubit} with the specified quantum state.
    \begin{verbatim}
        \qcmQ{<amplitude>:<phase>}{<amplitude>:<phase>}
    \end{verbatim}

    \item Draws a diagram for \textbf{two qubits} with the given specified quantum state.
    \begin{verbatim}
        \qcmQQ{<amplitude>:<phase>}
            {<amplitude>:<phase>}
            {<amplitude>:<phase>}
            {<amplitude>:<phase>}
    \end{verbatim}

    \item  Draws a diagram for \textbf{three qubits} with the given specified quantum state.
    \begin{verbatim}
        \qcmQQQ{<amplitude>:<phase>}
            {<amplitude>:<phase>}
            {<amplitude>:<phase>}
            {<amplitude>:<phase>}
            {<amplitude>:<phase>}
            {<amplitude>:<phase>}
            {<amplitude>:<phase>}
            {<amplitude>:<phase>}
    \end{verbatim}

  
    \item Changes the scale of the generated diagrams (e.g., \verb|\qcmScale{3}| is a good starting point).
  
    \item Draws the wireframe of the transition diagram used in the quantum cube model.
        To be used inside a tikz environment.
    \begin{verbatim}
        \qcmWireframe{1} % for a single qubit
        \qcmWireframe{2} % for two qubits
        \qcmWireframe{3} % for three qubits
    \end{verbatim}
\end{description}

\section*{Usage Examples}

\textit{Examples will be added here — please provide code snippets and expected output visuals.}

\section*{License}


\section*{References}

\fullcite{justQuantumComputingCompact2022}

\end{document}
