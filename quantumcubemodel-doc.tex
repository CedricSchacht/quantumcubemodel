\documentclass{article}
\usepackage[a4paper,margin=1in]{geometry}
\usepackage{hyperref}
\usepackage{subcaption}
\usepackage{quantumcubemodel}
\usepackage{listings}
\usepackage[backend=biber, style=authortitle]{biblatex}
\addbibresource{quantumcubemodel-bib.bib}
\usepackage{parskip}
\usepackage{pdflscape}

\usepackage{adjustbox}

\title{Quantum Cube Model LaTeX Package (v0.2.0)}
\author{Cedric Schacht \\ \texttt{cedric.schacht@dhbw-stuttgart.de}}
\date{\today}

\newcommand{\inputexample}[2][0.8\textwidth]{
  \lstinputlisting{examples/#2.tex}
  \input{examples/#2.tex}
}

\begin{document}
\maketitle

\section*{Introduction}

The \textbf{Quantum Cube Model (QCM)} package provides commands that make it easy to create diagrams representing the quantum cube model for up to 3 qubits.
The package simplifies drawing complex quantum state diagrams inspired by Prof.~B.~Just's framework.

The key difference to version v0.1.0 of this package is,
the new version exposes the tikz picture canves to the user.
This allows for far more refined diagramms, and flexability.

\section*{Requirements}
To use this package include \verb|\usepackage{quantumcubemodel}| in your documents preamble.
This package depends on the following LaTeX packages:
\begin{verbatim}
\RequirePackage{kvoptions}
\RequirePackage{braket}
\RequirePackage{xcolor}
\RequirePackage{tikz}
\usetikzlibrary{3d, calc, arrows.meta}
\end{verbatim}

\section*{License}
\begin{verbatim}
MIT License
Copyright (c) 2025 Cedric Schacht

Permission is hereby granted, free of charge, to any person obtaining a copy
of this software and associated documentation files (the "Software"), to deal
in the Software without restriction, including without limitation the rights
to use, copy, modify, merge, publish, distribute, sublicense, and/or sell
copies of the Software, and to permit persons to whom the Software is
furnished to do so, subject to the following conditions:

The above copyright notice and this permission notice shall be included in all
copies or substantial portions of the Software.

THE SOFTWARE IS PROVIDED "AS IS", WITHOUT WARRANTY OF ANY KIND, EXPRESS OR
IMPLIED, INCLUDING BUT NOT LIMITED TO THE WARRANTIES OF MERCHANTABILITY,
FITNESS FOR A PARTICULAR PURPOSE AND NONINFRINGEMENT. IN NO EVENT SHALL THE
AUTHORS OR COPYRIGHT HOLDERS BE LIABLE FOR ANY CLAIM, DAMAGES OR OTHER
LIABILITY, WHETHER IN AN ACTION OF CONTRACT, TORT OR OTHERWISE, ARISING FROM,
OUT OF OR IN CONNECTION WITH THE SOFTWARE OR THE USE OR OTHER DEALINGS IN THE
SOFTWARE.
\end{verbatim}

\section*{References}
\fullcite{justQuantumComputingCompact2022}

\section*{Changelog}
\begin{itemize}
  \item Provide qcmx environment that exposes tikz canvas.
  \item Give an option to set the main color.
  \item Provide commands for easier amplitude configuration.
  \item Provide options to change qubit axis.
  \item Enable the use of multiple qcm diagramms in one tikz canvas.
\end{itemize}

\section*{Package options}
To change the appearence of your diagramms you can set the main color as follows:
\inputexample{example-change-main-color}

\clearpage
\section*{Provided commands}
All commands are prefixed with \textbf{qcmx}.

There are other commands that start with \textbf{QCMXINTERNAL} these are for internal purposes only,
the use of those is not recomended.

\begin{verbatim}
  \begin{qcmx}
    % Everything happens in this environment
  \end{qcmx}
\end{verbatim}
In this environment you can call one of the following commands to render a qcm diagram 
for one, two or three qubit diagramms respectively:
\begin{itemize}
  \item \verb|\qcmxRenderQ|
  \item \verb|\qcmxRenderQQ|
  \item \verb|\qcmxRenderQQQ|
\end{itemize}

Initially all amplitudes are set to 0,
you can change them with the commands of the following pattern:

\verb|\qcmxO{}| or \verb|\qcmxI{}| for one qubit amplitued at $\ket{0}$ (O) or $\ket{1}$ (I).

\verb|\qcmxOO{}|, \verb|\qcmxOI{}|, \verb|\qcmxIO{}| and \verb|\qcmxII{}| for systems with two qubits.

\verb|\qcmxOOO{}| up to \verb|\qcmxIII{}| for systems with three qubits..

You have to set the amplitudes befor calling the corresponding render command.

Each of theses commands gives you an optional parameter to set a phase (in degrees from 0 to 360) for the coefficient,
for the state $-\ket{+}$ this results in:
\begin{verbatim}
  \begin{qcmx}
    \qcmxO[180]{0.71}
    \qcmxI[180]{0.71}
    \qcmxRenderQ
  \end{qcmx}
\end{verbatim}

\clearpage
\section*{Examples}
\inputexample{example-rendering-one-qubit}
\inputexample{example-rendering-one-qubit-arbitrary-states}

\clearpage
\inputexample[\textwidth]{example-rendering-phi-times-psi-equals-phipsi}



\input{examples/example-redering-one-qubit-hadamard.tex}
\end{document}